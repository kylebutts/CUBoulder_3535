\documentclass[11pt]{article}
\title{Writing Assignment Examples}
%% Language and font encodings
\usepackage[english]{babel}
\usepackage[utf8x]{inputenc}
\usepackage[T1]{fontenc}

\usepackage{helvet}

%% Sets page size and margins
\usepackage[letterpaper,top=3cm,bottom=2cm,left=3cm,right=3cm,marginparwidth=1.75cm]{geometry}

%% Useful packages
\usepackage{amsmath}
\usepackage{graphicx}
\usepackage{tcolorbox}
\usepackage{amssymb}
\usepackage{amsthm}
\usepackage{lastpage}
\usepackage{accents}
\usepackage{multicol}

% For better list numbering
\usepackage[shortlabels]{enumitem}

% Font
% \usepackage{tgbonum}


% Tikz
\usepackage{tikz}

\usetikzlibrary{calc,fit,shapes.misc,backgrounds}
\usepackage{pgfplots}
\pgfplotsset{compat = newest}
\usetikzlibrary{positioning, arrows.meta}
\usepgfplotslibrary{fillbetween}

% Headers
\usepackage{fancyhdr}
\pagestyle{fancy}

% Store \@title as \thetitle
\makeatletter
\let\thetitle\@title
\makeatother

\fancyhf{}
\lhead{\fontfamily{qbk}\fontsize{10}{11}\selectfont ECON 3535}
\rhead{\fontfamily{qbk}\fontsize{10}{11}\selectfont \thetitle}
\rfoot{\fontfamily{qbk}\fontsize{10}{11}\selectfont \thepage}


% Sections and Subsections

% define colors
\definecolor{buff-gold}{HTML}{CFB87C}
\definecolor{buff-grey}{HTML}{565A5C}
% custom tcolorbox
\tcbset{colframe=buff-gold, colback=white!100!black}

% new page per section
\usepackage{titlesec}
\newcommand{\sectionbreak}{\clearpage}
% change style of section
\usepackage{sectsty}
\sectionfont{\color{buff-gold} \fontfamily{qbk}\selectfont}
\subsectionfont{\color{buff-grey} \fontfamily{qbk}\selectfont}
\subsubsectionfont{\color{buff-grey} \fontfamily{qbk}\selectfont}



\newtoggle{INCLUDEANSWERS}
\toggletrue{INCLUDEANSWERS}
% \togglefalse{INCLUDEANSWERS}

\newcommand{\answer}[1]{\iftoggle{INCLUDEANSWERS}{{\color{violet!70!white}\textbf{Solution:} #1}}{}}

\newtoggle{INCLUDEPOINTS}
\toggletrue{INCLUDEPOINTS}
\newcommand{\points}[1]{\iftoggle{INCLUDEPOINTS}{{\color{blue!70!white}(#1 pts.)}}{}}

\usepackage{xurl}

\begin{document}
\section*{Example 1}

\noindent
The Reef Restoration Proposal

\noindent
To: Governor of Hawaii

\noindent
From: First Last

\noindent
Date: 4/10/2022

\subsection*{The Problem at Hand:}

All over the world as well as here in the United States fish populations, specifically those living within coral reefs, have been declining. This is the result of many things including but not limited to Global Warming, Overfishing, and Aquarium Collecting (A process in which fisherman will capture reef fish in order to sell them in the pet market). In this proposal however, we will specifically be looking at the effects of Aquarium Collecting, also known as "Reef Capture," which is an issue not many people know about. For example, according to research done by Brian Tissot and Leon Hallacher done in 2003 reef capture was shown to decrease 7 of the 10 studied fish populations by 40-75\%.

This project also included observational notes on the negative effects capturing had on the health of the reefs themselves. It should also be noted that ornamental fish are by and large one of the most valuable fish when looking at a per pound basis, meaning that there is a very valuable market at play here, and it's important that the demand is sated in a responsible manner. Some good news is that there have been attempts by government bodies to combat the destruction of their environment. Just recently Hawaii banned the collection of reef fish on all of their islands, as confirmed by The Center for Biological Diversity, which is where this proposal comes into play.

\subsection*{My Proposal:}

My proposal is one that would incentivize the business sector to take part in the restoration of reefs not only in the United States, but potentially the rest of the world. This feat can be accomplished through inland fish farms. A niche industry that is soon to become prevalent. For years most major fish farms have been conducted on our oceans shores, however, there are too many unforeseeable problems with this method such as poor weather and pollutants in the water supply. There is too much room for error. However, if we were to provide new inland fish farms with government grants as well as subsidies then we can bring about a more sustainable solution to the aquarium market. These subsidies would mostly be seen in water supply, energy supply, and wastewater payments. Larger scale inland fish farms have struggled to come to market mostly due to these three limitations that shore based farms do not see as much of.

\subsection*{Groups of Interest:}

The groups that will be most directly affected by my proposal are the aquaculture sector, tourists, locals, and the general population. The aquaculture sector has been described in detail above, and tourists are an obvious factor as those visiting our reefs would prefer to not have them desolated, but there is major economic value in our reefs that an everyday person might not appreciate. Jane Lunchenco, Ph.D., an NOAA (National Oceanic and Atmospheric Administration) administrator, stated that the NOAA study which found that the economic value of Hawaii's coral reefs is \$33.57 Billion "illustrates the economic value of coral reefs to all Americans, and how important it is to conserve these ecosystems for future generations." Our reefs are a not so small part of our national economy, and a huge part of Hawaii's. That is why not only does this proposal affect the aquaculture and tourist sectors, but also the lives of everyday Americans. There is also something to be said about the pride of the people who live around our world's coral reefs. Many local communities, such as that of Hawaii, feel that they have been extorted by the pet trade. Increasing the sentiment of the locals as well as helping to put the industry back into their hands is something to be considered as well. 

\subsection*{Overview:} 

The ban of reef capture by Hawaii is a fantastic start to the restoration of our reefs, however, how much of the taxpayers money can realistically go towards environmental preservation? In a perspective article written by Shawna Foo and Gregory Asner out of the Center for Global Discovery and Conservation Science department at Arizona State, it was stated that "Coral reef restoration is expensive, with costs estimated from \$1,717 to \$2,879,773 per hectare (A hectare equating to 10,000 square meters)." Hawaii's Coral Reefs alone equate to 165,921 hectares and if they were all being damaged would cost the government anywhere between \$285 million to \$478 billion dollars. This is not a feasible amount for the government to pay, so providing incentives for the business sector to partake in this process seems like a no-brainer seeing as it would be a fraction of the cost while the business sector could multiply the dollar amount put into it. For example, I interviewed Stephen Barron on rough estimates of his operating costs for Eco Harvest LLC, his new aquarium fish farm in Hawaii. Here is what they look like:
\begin{itemize}
  \item Energy Costs: \$0.33 kwH or \$200,000/yr
  \item Water Costs: \$200,000/yr
  \item Operating: \$2,000,000/yr
  \item Facility: \$3,500,000
\end{itemize}

Seeing as these numbers pale in comparison to full reef restoration costs we can see the efficacy the business sector operates at. This facility would be able to produce upwards of 450,000 yellow tangs a year. Many of the fish are to be exported in the pet trade, but some are to be released into the oceans in compliance with the Hawaiian government. In addition to the sentiment of the released fish Stephen added that "400,000 exported fish means that 400,000 fish that would've been captured will get to stay." Subsidies providing cheaper energy and water costs could increase these numbers as well considering the incentive for expansion they would provide. Stephen also mentioned that government grants are hugely important when looking at risk of investment. If my policy is upheld then businesses looking to get started can provide investors with additional credibility through means of a government grant, and if you look closely the government is not a huge partaker in commercial strategy when selecting grant candidates. Most of them go into the world of academia, which is great, but perhaps it is worth looking into providing the means of action needed to act on research.

Alternatives to this policy would either be paying for restoration efforts directly through the government or to use the money provided into research on reef restoration. I do not think it is in the best interest of anyone to directly fund restoration, however, using the money that would be used for this policy for research purposes could prove to be just as monetarily effective. If the funding for research would provide a means of cheaper restoration then it would be extremely competitive with this proposal. The only concern here is there is no guarantee of that ever happening, while up and coming businesses provide a sense of tangibility that research alone can only provide if successful.

Finally, a negative externality of these incentives would most likely fall onto the local populations. At first glance it seems as if it would benefit the locals, but if we hypothesize what could happen we see potential issues. For example, while Stephen himself insists on hiring locals to operate his facility there are those that may not think the same way. Businesses could come to these reef locations, set up shop, and operate it through their own independent means. Although these businesses would most likely provide a net positive to the overall health of a reef economy, it could end up hurting opportunities that could be provided to individuals who are native to the area. Perhaps a solution to this would be to make grants and subsidies come with a condition of local hires, although I am not sure how effectively this could be accomplished. 

\subsection*{Conclusion:}

I propose that reef communities within the United States provide hatcheries and inland fish farms the opportunity for success through means of government grants and subsidies. These incentives will allow the commercial industry to become a bigger part of environmental restoration efforts. In addition to restoration these businesses will provide an influx of capital into local reef communities as well as into the aquaculture industry. These businesses can accomplish such feats at a higher efficiency than direct payments through the government. Thank you for listening to my proposal, and hopefully with it we can create a healthier environment that provides substantial monetary opportunities.

\subsection*{References:}

\smallskip\noindent Stephen Barron - COO of EcoHarvest LLC

\smallskip\noindent Center for Biological Diversity. "Ban on Collecting Hawaii Reef Fish for Aquarium Trade Upheld."

\smallskip\noindent Center for Biological Diversity, Center for Biological Diversity, 14 Aug. 2020,\url{https://biologicaldiversity.org/w/news/press-releases/ban-collecting-hawaii-reef-fish-aquarium-trade-upheld-2020-08-14/}. The Society for Conservation Biology. \url{https://conbio.onlinelibrary.wiley.com/doi/10.1111/j.1523-1739.2003.00379.x}.

\smallskip\noindent US Department of Commerce, National Oceanic and Atmospheric Administration. "Total Economic Value for Protecting and Restoring Hawaiian Coral Reef Ecosystems." NOAA Coral Reef Information System (CoRIS) Home Page, 12 Oct. 2007, \url{https://www.coris.noaa.gov/activities/hawaii_econeval/}.

\smallskip\noindent Foo, Shawna A., and Gregory P. Asner. "Scaling up Coral Reef Restoration Using Remote Sensing Technology." Frontiers, Frontiers, 1 Jan. 1AD, \url{https://www.frontiersin.org/articles/10.3389/fmars.2019.00079/full}.


\newpage 
\section*{Example 2}

\noindent
The Reef Restoration Proposal

\noindent
To: Governor of California

\noindent
From: First Last

\noindent
Date: 4/10/2022

\subsection*{Introduction}

This memo addresses carbon offsets in California's carbon cap-and-trade program. I will summarize California's intentions with the policy and explain how it works. I will then identify groups of interest, trade-offs, and costs and benefits inherent to the policy. Lastly, I will discuss flaws in the policy regarding the valuation of forest offset credits and environmental justice issues.

\subsection*{California's Carbon Cap-and-Trade Program}

In 2013, California launched an Emissions Trade System (ETS) to reduce greenhouse gas emissions. The state has goals of reducing its emissions to 40\% below 1990 levels by 2030 and 80\% below 1990 levels by 2050. California is aiming for economy-wide carbon neutrality by 2045. The cap-and-trade program plays an essential role in meeting these goals. The businesses enrolled in the program are responsible for about 85\% of California's total emissions (California Air Resources Board (2017)). According to California's 2017 Climate Change Scoping Plan, the cap-and-trade system will achieve about 38\% of total cumulative emissions reductions between 2021 and 2030 (Center for Climate and Energy Solutions (2021)). California directs 35\% of the revenue towards environmentally disadvantaged and low-income communities (Center for Climate and Energy Solutions (2021)). The California Air Resources Board (CARB) announced on November 3, 2021, that they achieved a 100\% compliance rate in 2018-2020 from the corporations in the cap-and-trade program (California Air Resources Board (2021)).


\subsection*{The Compliance Offset Program} 

The Compliance Offset Program allows polluters to buy credits that offset their carbon emissions. Essentially, companies have three options: lower emissions, buy permits to keep emitting, or offset emissions. The California Air Resources Board issues CARB offset credits to projects that reduce greenhouse gases, usually by improved forest management (IFM). Offset credits can account for 4\% of total compliance obligation between 2021 and 2025 and 6\% between 2026 and 2030. In 2021, CARB mandated that at least half of the offset projects must directly benefit California (California Air Resources Board (2022)).

\subsection*{Groups of Interest}

The Offset Program in California has multiple groups of interest. The polluting corporations are the buyers of these offset credits, and the sellers are the offset projects, including IFM projects. While buyers are in California, sellers can be located anywhere in the United States. This relationship shifts the benefits of the improved carbon sinks out of California while the costs of continued pollution stay in-state. Local communities close to polluting businesses and local communities close to the offset projects have an interest in the trade of credits. The state of California has an interest in lowering its carbon emissions and making revenue from the program. As with all greenhouse gases, everyone benefits globally by California reducing its carbon emissions, but these benefits cannot be directly traced to a specific location or population.

\subsection*{Trade-offs}

In California's cap-and-trade program, the trade-off is choosing whether to abate, pay, or offset. The alternative policy to a cap-and-trade system is a Pigouvian carbon tax. If they are done right, cap-and-trade and a carbon tax should lead to the same result in the cheapest way possible. They both provide the same trade-off of choosing to abate or pay the allowance/tax, generating government revenue. Both policies provide incentives to develop low-carbon technologies and ensure that abatement is done in the cheapest way possible. When carbon offset credits are introduced to the trade, the valuation of forest carbon sequestration and costs of lowering pollution must relate to each other. Carbon offset credits are based on 'what if' scenarios that are difficult to value with the needed precision. A recent report showed that corporations 'banked' several allowances from the 2018-2020 period (Burtraw et al. (2021)). Correct allocation of allowances is crucial to the success of a cap-and-trade system, but it is also a challenging task. Over-allocating allowances jeopardize California's future emission goals and cause inefficient outcomes.

\subsection*{Costs and Benefits}

The cost of California's cap-and-trade system is estimated to be \$1.6 billion - \$5.1 billion (in \$2015) from 2021 to 2030. The cost depends on the allowance prices, which can fluctuate between \$16.2 - \$72.9 in 2021 and \$25.2 - \$81.9 in 2030 (California Air Resources Board (2017)). In California's 2017 Climate Scoping Plan, they also estimated benefits from the cap-and-trade system by calculating the social costs that were avoided by the program in its efforts to cut down emissions. The avoided social cost of carbon (in \$2015) is \$0.6 billion-\$6.5 billion in 2021-2030. Offset credits provide economic and environmental benefits to landowners, such as foresters and farmers. They have incentives to preserve carbon sinks that they would have taken advantage of without the program. These incentives are essential to mitigate greenhouse gas emissions from land-use change. The IPCC estimated that 23\% of anthropogenic greenhouse gas emissions (2007-2016) derive from land-use change (IPCC (2019)). Costs from the offset program can be found in the valuation methods of offset projects. If future carbon emission mitigation from these projects is not measured correctly, corporations can potentially emit greenhouse gases that are not accounted for by their offset credits.


\subsection*{Systematic Over-Crediting of Offset Projects}

In a research study by Badgley et al. (2022), they found that the forest carbon offset credits were systematically over-valued. They concluded that California's offset program creates incentives to exaggerate the amount of carbon a project could offset so that the credits don't reflect real climate benefits. Haya et al. estimated that 29.4\% of the projects were over-credited, equivalent to 30.0 million tCO2e out of the 102.1 million tCO2e worth of upfront credits. This conclusion is based on a comparison of CARB numbers of common practice and the researchers' calculated numbers of common practice.\footnote{Common practice is an estimate of tCO2e/acre sequestered in the forest without offset carbon payments. Offset credits are meant to keep a certain number above common practice. If standard practice is calculated as too low, projects can sell more credits.} This error was mainly due to the way that CARB categorizes forest types which made it possible for projects to enroll themselves as a forest type that sequestered less carbon and thereby lower its common practice estimate (Badgley et al. (2022)).

Carbon offset credits are based on uncertain future scenarios that are difficult to measure precisely and value. With an incentive to over-credit projects from sellers, benefits from forest management do not equal costs from continued pollution. Furthermore, IFM offset credits are measured over 100 years, but carbon dioxide has an atmospheric lifetime of 300-1,000 years (Buis (2019)). Even though we might not be able to measure the benefits of forest management that far into the future, the discrepancy should be considered since long-term sustainability and carbon neutrality is California's goal.

\subsection*{Environmental Injustice}

Local pollutants are not accounted for when businesses offset their carbon emissions. The ones who pay the costs of local pollutants are communities close to major polluters. A study from the University of Southern California found that communities with predominantly people of color, people living below the poverty level, and people with less education are more likely to live near cap-and-trade facilities. They are also less likely to experience a reduction in pollution from local facilities (Pastor et al. (2022)). Offset credits pose a critical dilemma: whom do we want to reap the benefits? Chevron owns refineries in marginalized communities in Richmond and El Segundo but has bought offset credits from the Passamaquoddy tribe in the forests of Maine. The tribe has suffered from environmental injustice daily, and the offsets have helped them improve blueberry harvesting, maple syrup production businesses and support members who struggle with opioid addiction. However, these benefits were not free: communities close to the refineries continuously battle with health conditions that stem from low air quality. The point of a cap-and-trade program or a carbon tax is to internalize external costs to reflect the social cost of polluting, but California's program still imposes costs of polluting on local, disadvantaged communities (Halper (2021)).


\subsection*{Conclusion}

The Offset Compliance Program provides incentives to invest in improved forest management and other offset projects that positively affect the global environment and the local communities. The cap-and-trade policy has been successful in meeting California's emission goals at a low cost. It has also made government revenue that has been reinvested in environmentally disadvantaged communities. Major critiques of the policy include the over-valuation of forest offset credits and environmental injustice for local communities. Forest offset projects are systemically overcredited because there is an incentive from landowners to exaggerate the positive effects of the projects. Local communities close to the offset projects benefit at the expense of local communities to polluting corporations in California. Pollution in disadvantaged communities was reduced less than pollution in other communities under the cap-and-trade program. Carbon offsets do not account for local pollutants that can have lasting negative health effects. The offset program is inefficient when offset projects are used towards reaching an emissions goal.



\subsection*{References}

\smallskip\noindent
100\% of companies in cap-and-trade program meet 2020 compliance obligations | California Air Resources Board. (2021, November 3). 

\smallskip\noindent
California Air Resources Board. \url{https://ww2.arb.ca.gov/news/100-companies-cap-and-trade-program-meet-2020-complianceobligations}. Retrieved April 10, 2022.

\smallskip\noindent
Badgley, G., Freeman, J., Hamman, J. J., Haya, B., Trugman, A. T., Anderegg, W. R. L., \& Cullenward, D. (2022). Systematic over--crediting in California's forest carbon offsets program. Global Change Biology, 28(4), 1433--1445. \url{https://doi.org/10.1111/gcb.15943}. Retrieved April 10, 2022.

\smallskip\noindent
Buis, A. (2019, October 9). The Atmosphere: Getting a Handle on Carbon Dioxide. Climate Change: Vital Signs of the Planet; NASA. \url{https://climate.nasa.gov/news/2915/the-atmosphere-getting-ahandle-on-carbon-dioxide}. Retrieved April 10, 2022.

\smallskip\noindent
Burtraw, D., Cullenward, D., Fowlie, M., Sutter, K., \& Brown, R. (2022). 2021 Annual Report of the Independent Emissions Market Advisory Committee. California Environmental Protection Agency. \url{https://sbud.senate.ca.gov/sites/sbud.senate.ca.gov/files/2021-IEMAC-Annual-Report.pdf}. Retrieved April 10, 2022.

\smallskip\noindent
California Cap and Trade. (2021, August 24). Center for Climate and Energy Solutions. \url{https://www.c2es.org/content/california-cap-and-trade/}. Retrieved April 10, 2022.

\smallskip\noindent
California's 2017 Climate Change Scoping Plan. (2017). California Air Resources Board. \url{https://ww2.arb.ca.gov/sites/default/files/classic/cc/scopingplan/scoping_plan_2017.pdf}. Retrieved April 10, 2022.

\smallskip\noindent
Compliance Offset Program | California Air Resources Board. (2022). California Air Resources Board. \url{https://ww2.arb.ca.gov/our-work/programs/compliance-offset-program/about}. Retrieved April 10, 2022.

\smallskip\noindent
Halper, E. (2021, September 8). Burned trees and billions in cash: How a California climate program lets companies keep polluting. Los Angeles Times. \url{https://www.latimes.com/politics/story/202109-08/what-is-the-california-climate-credit-does-it-cut-pollution}. Retrieved April 10, 2022.

\smallskip\noindent
Haya, B., Cullenward, D., Strong, A. L., Grubert, E., Heilmayr, R., Sivas, D. A., \& Wara, M. (2020). Managing uncertainty in carbon offsets: Insights from California's standardized approach. Climate Policy, 20(9), 1112--1126. \url{https://doi.org/10.1080/14693062.2020.1781035}. Retrieved April 10, 2022.

\smallskip\noindent
IPCC Special Report on Climate Change, Desertification, Land Degradation, Sustainable Land Management, Food Security, and Greenhouse gas fluxes in Terrestrial Ecosystems. (2019). IPCC. \url{https://www.ipcc.ch/site/assets/uploads/2019/08/4.SPM_Approved_Microsite_FINAL.pdf}. Retrieved April 10, 2022.


\smallskip\noindent
Pastor, M., Ash, M., Cushing, L., Morello-Frosch, R., Muña, E.-M., \& Sadd, J. (2022). Up in the Air: Revisiting Equity Dimensions of California's Cap-and-Trade System. USC Dornsife Equity Research Institute. \url{https://dornsife.usc.edu/assets/sites/1411/docs/CAP_and_TRADE_Updated_2020_v02152022_FINAL.pdf}. Retrieved April 10, 2022.







\end{document}
