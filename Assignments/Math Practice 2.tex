\documentclass[11pt]{article}
\title{Math Practice 2}
%% Language and font encodings
\usepackage[english]{babel}
\usepackage[utf8x]{inputenc}
\usepackage[T1]{fontenc}

\usepackage{helvet}

%% Sets page size and margins
\usepackage[letterpaper,top=3cm,bottom=2cm,left=3cm,right=3cm,marginparwidth=1.75cm]{geometry}

%% Useful packages
\usepackage{amsmath}
\usepackage{graphicx}
\usepackage{tcolorbox}
\usepackage{amssymb}
\usepackage{amsthm}
\usepackage{lastpage}
\usepackage{accents}
\usepackage{multicol}

% For better list numbering
\usepackage[shortlabels]{enumitem}

% Font
% \usepackage{tgbonum}


% Tikz
\usepackage{tikz}

\usetikzlibrary{calc,fit,shapes.misc,backgrounds}
\usepackage{pgfplots}
\pgfplotsset{compat = newest}
\usetikzlibrary{positioning, arrows.meta}
\usepgfplotslibrary{fillbetween}

% Headers
\usepackage{fancyhdr}
\pagestyle{fancy}

% Store \@title as \thetitle
\makeatletter
\let\thetitle\@title
\makeatother

\fancyhf{}
\lhead{\fontfamily{qbk}\fontsize{10}{11}\selectfont ECON 3535}
\rhead{\fontfamily{qbk}\fontsize{10}{11}\selectfont \thetitle}
\rfoot{\fontfamily{qbk}\fontsize{10}{11}\selectfont \thepage}


% Sections and Subsections

% define colors
\definecolor{buff-gold}{HTML}{CFB87C}
\definecolor{buff-grey}{HTML}{565A5C}
% custom tcolorbox
\tcbset{colframe=buff-gold, colback=white!100!black}

% new page per section
\usepackage{titlesec}
\newcommand{\sectionbreak}{\clearpage}
% change style of section
\usepackage{sectsty}
\sectionfont{\color{buff-gold} \fontfamily{qbk}\selectfont}
\subsectionfont{\color{buff-grey} \fontfamily{qbk}\selectfont}
\subsubsectionfont{\color{buff-grey} \fontfamily{qbk}\selectfont}



\newtoggle{INCLUDEANSWERS}
% \toggletrue{INCLUDEANSWERS}
\togglefalse{INCLUDEANSWERS}

\newcommand{\answer}[1]{\iftoggle{INCLUDEANSWERS}{{\color{violet!70!white}\textbf{Solution:} #1}}{}}

\newtoggle{INCLUDEPOINTS}
\toggletrue{INCLUDEPOINTS}
\newcommand{\points}[1]{\iftoggle{INCLUDEPOINTS}{{\color{blue!70!white}(#1 pts.)}}{}}


\begin{document}


\subsection*{Tradable Permits}

\begin{enumerate}
  \item Two firms can control emissions at the following marginal costs: $MC_x = 80a_x$ and $MC_y = 40 a_y$ where $a_x$ and $a_y$ are, respectively, the amount of emissions reduced by firm $x$ and firm $y$. Assume that with no control at all, each firm would be emitting $50$ units of emissions or a total of $100$ units for both firms.

  \begin{enumerate}
    \item Which firm is better at abating pollution?
    
    \item If the goal is to reduce total emissions to $60$ units. How many units must be abated? Write out the abatement constraint in mathematical terms
    
    \item Consider a uniform standard. How many units must be abated by both firms? How much did each firm have to pay to abate their marginal unit of pollution?
  
    \item Consider a cap-and-trade system that aims for a total $60$ units of emissions. 
    \begin{enumerate}
      \item In words, describe why the marginal abatement costs for each firm must be equal to eachother in order to be at equilibrium (the optimality condition). 
      
      \item Using the optimality condition and the abatement constraint, solve for the equilibrium allocation of permits to each firm?
  
      \item At what price would these permits sell for at an auction?
    \end{enumerate}
  
    \item Assume that the control authority wanted to reach its objective by using an emissions charge system instead.
    \begin{enumerate}
      \item What tax amount should them impose to reach this equilibrium?
      
      \item How much revenue would the government collect?
    \end{enumerate}
  
    \item Why is cap-and-trade more cost-effective than a uniform standard where each firm reduces pollution by the same amount?
  \end{enumerate}

  \answer{
    \begin{enumerate}
      \item Firm $y$ has a lower cost of abatement, so it is the `better' firm at abating. 
            
      \item Going from 100 units to 60 units of pollution implies an abatement goal of 40 units. In mathematical notation, we have
      $$
        a_x + a_y = 40.
      $$
      
      \item Each firm needs to abate half, so $a_x = a_y = 20$. For the 20th unit, we have $MAC_x = 80 * 20 = 1600$ and $MAC_y = 40 * 20 = 800$. In words, the 20th unit cost firm $x$ \$1600 to abate and firm $y$ \$800 to abate.
    
      \item Consider a cap-and-trade system that aims for a total $60$ units of emissions. 
      \begin{enumerate}
        \item The marginal abatement costs must be equal to each other in order to ensure that the total cost of abatement is as cheap as possible. Consider the case where $(a_x, a_y)$ is such that $MAC_x(a_x) > MAC_y(a_y)$ and $a_x + a_y = 40$. We can lower cost by having firm $y$ abate one more unit, spending $MAC_y(a_y)$ dollars and having firm $x$ abate one fewer unit, saving $MAC_x(a_x)$. The abatement goal will still be hit, but costs will be cheaper since savings $MAC_x(a_x)$ are bigger than additional costs $MAC_y(a_y)$. The same will hold with $MAC_y(a_y)> MAC_x(a_x)$. Therefore, we need equality to hold in order for the abatement goal to be achieved at as low of a cost as possible. 
        
        \item Our optimality condition is $80 a_x = 40 a_y$ which implies $2 a_x = a_y$. Plugging into our constraint yields $a_x + 2 a_x = 40 \implies a_x^* = 40/3 \approx 13.33$. This implies $a_y^* = 40 - 40/3 = 80/3 \approx 26.67$.
    
        \item The permit price is equal to $MAC_x(a_x^*) = MAC_y(a_y^*) = 80 * 40/3 = 3200/3 \approx \$ 1066.67$. 
      \end{enumerate}
    
      \item Assume that the control authority wanted to reach its objective by using an emissions charge system instead.
      \begin{enumerate}
        \item The tax amount should be equal to the permit price solved for above, so $T = \$1066.67$. 
        
        \item After the policy, the firms will polute 60 units, so the revenue equals $1066.67 * 60 \approx \$ 64,000$. 
      \end{enumerate}
    
      \item In the uniform standard, the firm with higher abatement costs is abating more than is optimal. The cap-and-trade system balances this trade-off by setting $MAC_x = MAC_y$ as required for efficiency as argued above.
    \end{enumerate}
  }

  \item Two firms can control emissions at the following marginal costs: $MC_x = 200 a_x$ and $MC_y = 100 a_y$ where $a_x$ and $a_y$ are, respectively, the amount of emissions reduced by firm $x$ and firm $y$. Assume that with no control at all, each firm would be emitting $20$ units of emissions or a total of $40$ units for both firms.

  \begin{enumerate}
    \item Consider a cap-and-trade system that aims for a total reduction of $21$ units of emissions is necessary. 
    \begin{enumerate}
      \item What is the equilibrium allocation of permits to each firm?
      \item At what price would these permits sell for at an auction
    \end{enumerate}

    \item Assume that the control authority wanted to reach its objective by using an emissions
    charge system instead.
    \begin{enumerate}
      \item What tax amount should them impose to reach this equilibrium?
      \item How much revenue would the government collect?
    \end{enumerate}

    \item Why is cap-and-trade more cost-effective than a uniform standard where each firm reduces pollution by $10.5$ units?
  \end{enumerate}

  \answer{
    \begin{enumerate}
      \item Our constaint is $a_x + a_y = 21$ and our optimality condition is $200 a_x = 100 a_y \implies 2 a_x = a_y$. 
      \begin{enumerate}
        \item Plugging the optimality condition into the constraint yields $a_x + 2 a_x = 21 \implies a_x^* = 7$. Plugging $a_x^*$ back into the budget constraint yields $a_y^* = 14$. 
        
        \item Permits would sell for $MC_x(a_x^*) = MC_y(a_y^*) = \$1400$. 
      \end{enumerate}
  
      \item Assume that the control authority wanted to reach its objective by using an emissions charge system instead.
      \begin{enumerate}
        \item The tax price should equal $\$1400$ to reach the same outcome. 
        \item The firms would pollute 19 units generating total government revenue of $19 * 1400 = \$ 26,600$. 
      \end{enumerate}
  
      \item When $a_x = a_y = 10.5$, we have $MAC_x(10.5) = 1050 < MAC_y(10.5) = 2100$. In this case, total costs are not minimized since firm $y$ has a higher marginal abatement cost than firm $x$.
    \end{enumerate}
  }

  \item Two firms can control emissions at the following marginal costs: $ MC_x = 5 + 10 a_x $ and $MC_y = 11 a_y$ where $a_x$ and $a_y$ are, respectively, the amount of emissions reduced by firm $x$ and firm $y$. Assume that with no control at all, each firm would be emitting $10$ units of emissions or a total of $20$ units for both firms.

  \begin{enumerate}
    \item Consider a cap-and-trade system that aims for a total reduction of $10$ units of emissions. 
    \begin{enumerate}
      \item What is the equilibrium allocation of permits to each firm?
      \item At what price would these permits sell for at an auction
    \end{enumerate}

    \item Assume that the control authority wanted to reach its objective by using an emissions charge system instead.
    \begin{enumerate}
      \item What tax amount should them impose to reach this equilibrium?
      \item How much revenue would the government collect?
    \end{enumerate}
  \end{enumerate}

  \answer{
    \begin{enumerate}
      \item Our constaint is $a_x + a_y = 10$ and our optimality condition is $5 + 10 a_x = 11 a_y$. 
      \begin{enumerate}
        \item Rewriting the constraint as $a_x = 10 - a_y$ and plugging it into the optimality condition yields
        $$
          5 + 10 * (10 - a_y) = 11a_y \implies 105 = 21a_y \implies a_y^* = 5.
        $$
        Plugging $a_y^*$ into the budget constraint yields $a_x^* = 5$. 
        
        \item Permits would sell for $MC_x(a_x^*) = MC_y(a_y^*) = \$55$. 
      \end{enumerate}
  
      \item Assume that the control authority wanted to reach its objective by using an emissions charge system instead.
      \begin{enumerate}
        \item The tax price should equal $\$55$ to reach the same outcome. 
        \item The firms would pollute 10 units generating total government revenue of $10 * 55 = \$ 550$. 
      \end{enumerate}
    \end{enumerate}
  }
\end{enumerate}

\end{document}
