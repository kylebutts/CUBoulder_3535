\documentclass[11pt]{article}
\title{Math Practice 2}
%% Language and font encodings
\usepackage[english]{babel}
\usepackage[utf8x]{inputenc}
\usepackage[T1]{fontenc}

\usepackage{helvet}

%% Sets page size and margins
\usepackage[letterpaper,top=3cm,bottom=2cm,left=3cm,right=3cm,marginparwidth=1.75cm]{geometry}

%% Useful packages
\usepackage{amsmath}
\usepackage{graphicx}
\usepackage{tcolorbox}
\usepackage{amssymb}
\usepackage{amsthm}
\usepackage{lastpage}
\usepackage{accents}
\usepackage{multicol}

% For better list numbering
\usepackage[shortlabels]{enumitem}

% Font
% \usepackage{tgbonum}


% Tikz
\usepackage{tikz}

\usetikzlibrary{calc,fit,shapes.misc,backgrounds}
\usepackage{pgfplots}
\pgfplotsset{compat = newest}
\usetikzlibrary{positioning, arrows.meta}
\usepgfplotslibrary{fillbetween}

% Headers
\usepackage{fancyhdr}
\pagestyle{fancy}

% Store \@title as \thetitle
\makeatletter
\let\thetitle\@title
\makeatother

\fancyhf{}
\lhead{\fontfamily{qbk}\fontsize{10}{11}\selectfont ECON 3535}
\rhead{\fontfamily{qbk}\fontsize{10}{11}\selectfont \thetitle}
\rfoot{\fontfamily{qbk}\fontsize{10}{11}\selectfont \thepage}


% Sections and Subsections

% define colors
\definecolor{buff-gold}{HTML}{CFB87C}
\definecolor{buff-grey}{HTML}{565A5C}
% custom tcolorbox
\tcbset{colframe=buff-gold, colback=white!100!black}

% new page per section
\usepackage{titlesec}
\newcommand{\sectionbreak}{\clearpage}
% change style of section
\usepackage{sectsty}
\sectionfont{\color{buff-gold} \fontfamily{qbk}\selectfont}
\subsectionfont{\color{buff-grey} \fontfamily{qbk}\selectfont}
\subsubsectionfont{\color{buff-grey} \fontfamily{qbk}\selectfont}



\newtoggle{INCLUDEANSWERS}
\toggletrue{INCLUDEANSWERS}
% \togglefalse{INCLUDEANSWERS}

\newcommand{\answer}[1]{\iftoggle{INCLUDEANSWERS}{{\color{violet!70!white}\textbf{Solution:} #1}}{}}

\newtoggle{INCLUDEPOINTS}
\toggletrue{INCLUDEPOINTS}
\newcommand{\points}[1]{\iftoggle{INCLUDEPOINTS}{{\color{blue!70!white}(#1 pts.)}}{}}


\begin{document}

\subsection*{Two period model of a non-renewable resource}

Consider extraction of a non-renewable natural resource. The inverse demand function for the depletable resource is $P = 12 - Q$ in both periods $1$ and $2$ and the marginal cost of supplying it is $\$3$. The discount rate is $10\%$. There are $7.5$ units total.

\begin{enumerate}
  \item Find the equilibrium allocation in each period for resource extraction

  \item Describe using the concept of marignal user cost why $Q_1 = 3$ and $Q_2 = 4.5$ is not optimal

  \item What is the marginal user cost? Interpret this number.

  \item Now assume $r = 0$. What is the optimal allocation now? Why did it change in the direction that it did?
\end{enumerate}

\subsection*{Two period model of a non-renewable resource}

Consider extraction of a non-renewable natural resource. The inverse demand function for the depletable resource is $P = 12 - Q$ in both periods $1$ and $2$ and the marginal cost of supplying it is $2 + Q/2$. The discount rate is $6\%$. There are $15\$$ units total.

\begin{enumerate}
  \item Find the equilibrium allocation in each period for resource extraction

  \item What is the marginal user cost? Interpret this number.
\end{enumerate}

\subsection*{Tradable Permits}

Two firms can control emissions at the following marginal costs: $MC_1 = 200 a_x$ and $MC_2 = 100 a_y$ where $a_x$ and $a_y$ are, respectively, the amount of emissions reduced by the first and second firms. Assume that with no control at all, each firm would be emitting $20$ units of emissions or a total of $40$ units for both firms.

\begin{enumerate}
  \item Consider a cap-and-trade system that aims for a total reduction of $21$ units of emissions is necessary. 
  \begin{enumerate}
    \item What is the equilibrium allocation of permits to each firm?
    \item At what price would these permits sell for at an auction
  \end{enumerate}

  \item Assume that the control authority wanted to reach its objective by using an emissions
  charge system instead.
  \begin{enumerate}
    \item What tax amount should them impose to reach this equilibrium?
    \item How much revenue would the government collect?
  \end{enumerate}

  \item Why is cap-and-trade more cost-effective than a uniform standard where each firm reduces pollution by $10.5$ units?
\end{enumerate}

\subsection*{Tradable Permits}


Two firms can control emissions at the following marginal costs: $ MC_1 = 5 + 10 a_x $ and $MC_2 = 11 a_y$ where $a_x$ and $a_y$ are, respectively, the amount of emissions reduced by the first and second firms. Assume that with no control at all, each firm would be emitting $20$ units of emissions or a total of $10$ units for both firms.

\begin{enumerate}
  \item Consider a cap-and-trade system that aims for a total reduction of $21$ units of emis-
  sions is necessary. 
  \begin{enumerate}
    \item What is the equilibrium allocation of permits to each firm?
    \item At what price would these permits sell for at an auction
  \end{enumerate}

  \item Assume that the control authority wanted to reach its objective by using an emissions
  charge system instead.
  \begin{enumerate}
    \item What tax amount should them impose to reach this equilibrium?
    \item How much revenue would the government collect?
  \end{enumerate}
\end{enumerate}


\end{document}
