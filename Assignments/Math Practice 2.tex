\documentclass[11pt]{article}
\title{Math Practice 2}
%% Language and font encodings
\usepackage[english]{babel}
\usepackage[utf8x]{inputenc}
\usepackage[T1]{fontenc}

\usepackage{helvet}

%% Sets page size and margins
\usepackage[letterpaper,top=3cm,bottom=2cm,left=3cm,right=3cm,marginparwidth=1.75cm]{geometry}

%% Useful packages
\usepackage{amsmath}
\usepackage{graphicx}
\usepackage{tcolorbox}
\usepackage{amssymb}
\usepackage{amsthm}
\usepackage{lastpage}
\usepackage{accents}
\usepackage{multicol}

% For better list numbering
\usepackage[shortlabels]{enumitem}

% Font
% \usepackage{tgbonum}


% Tikz
\usepackage{tikz}

\usetikzlibrary{calc,fit,shapes.misc,backgrounds}
\usepackage{pgfplots}
\pgfplotsset{compat = newest}
\usetikzlibrary{positioning, arrows.meta}
\usepgfplotslibrary{fillbetween}

% Headers
\usepackage{fancyhdr}
\pagestyle{fancy}

% Store \@title as \thetitle
\makeatletter
\let\thetitle\@title
\makeatother

\fancyhf{}
\lhead{\fontfamily{qbk}\fontsize{10}{11}\selectfont ECON 3535}
\rhead{\fontfamily{qbk}\fontsize{10}{11}\selectfont \thetitle}
\rfoot{\fontfamily{qbk}\fontsize{10}{11}\selectfont \thepage}


% Sections and Subsections

% define colors
\definecolor{buff-gold}{HTML}{CFB87C}
\definecolor{buff-grey}{HTML}{565A5C}
% custom tcolorbox
\tcbset{colframe=buff-gold, colback=white!100!black}

% new page per section
\usepackage{titlesec}
\newcommand{\sectionbreak}{\clearpage}
% change style of section
\usepackage{sectsty}
\sectionfont{\color{buff-gold} \fontfamily{qbk}\selectfont}
\subsectionfont{\color{buff-grey} \fontfamily{qbk}\selectfont}
\subsubsectionfont{\color{buff-grey} \fontfamily{qbk}\selectfont}



\newtoggle{INCLUDEANSWERS}
\toggletrue{INCLUDEANSWERS}
% \togglefalse{INCLUDEANSWERS}

\newcommand{\answer}[1]{\iftoggle{INCLUDEANSWERS}{{\color{violet!70!white}\textbf{Solution:} #1}}{}}

\newtoggle{INCLUDEPOINTS}
\toggletrue{INCLUDEPOINTS}
\newcommand{\points}[1]{\iftoggle{INCLUDEPOINTS}{{\color{blue!70!white}(#1 pts.)}}{}}


\begin{document}


\subsection*{Tradable Permits}

\begin{enumerate}
  \item Two firms can control emissions at the following marginal costs: $MC_1 = 80a_x$ and $MC_2 = 40 a_y$ where $a_x$ and $a_y$ are, respectively, the amount of emissions reduced by the first and second firms. Assume that with no control at all, each firm would be emitting $50$ units of emissions or a total of $100$ units for both firms.

  \begin{enumerate}
    \item Which firm is better at abating pollution?
    
    \item If the goal is to reduce total emissions to $60$ units. How many units must be abated? Write out the abatement constraint in mathematical terms
    
    \item Consider a uniform standard. How many units must be abated by both firms? How much did each firm have to pay to abate their marginal unit of pollution?
  
    \item Consider a cap-and-trade system that aims for a total $60$ units of emissions. 
    \begin{enumerate}
      \item In words, describe why the marginal abatement costs for each firm must be equal to eachother in order to be at equilibrium (the optimality condition). 
      
      \item Using the optimality condition and the abatement constraint, solve for the equilibrium allocation of permits to each firm?
  
      \item At what price would these permits sell for at an auction?
    \end{enumerate}
  
    \item Assume that the control authority wanted to reach its objective by using an emissions charge system instead.
    \begin{enumerate}
      \item What tax amount should them impose to reach this equilibrium?
      
      \item How much revenue would the government collect?
    \end{enumerate}
  
    \item Why is cap-and-trade more cost-effective than a uniform standard where each firm reduces pollution by the same amount?
  \end{enumerate}

  \item Two firms can control emissions at the following marginal costs: $MC_1 = 200 a_x$ and $MC_2 = 100 a_y$ where $a_x$ and $a_y$ are, respectively, the amount of emissions reduced by the first and second firms. Assume that with no control at all, each firm would be emitting $20$ units of emissions or a total of $40$ units for both firms.

  \begin{enumerate}
    \item Consider a cap-and-trade system that aims for a total reduction of $21$ units of emissions is necessary. 
    \begin{enumerate}
      \item What is the equilibrium allocation of permits to each firm?
      \item At what price would these permits sell for at an auction
    \end{enumerate}

    \item Assume that the control authority wanted to reach its objective by using an emissions
    charge system instead.
    \begin{enumerate}
      \item What tax amount should them impose to reach this equilibrium?
      \item How much revenue would the government collect?
    \end{enumerate}

    \item Why is cap-and-trade more cost-effective than a uniform standard where each firm reduces pollution by $10.5$ units?
  \end{enumerate}

  \item Two firms can control emissions at the following marginal costs: $ MC_1 = 5 + 10 a_x $ and $MC_2 = 11 a_y$ where $a_x$ and $a_y$ are, respectively, the amount of emissions reduced by the first and second firms. Assume that with no control at all, each firm would be emitting $20$ units of emissions or a total of $10$ units for both firms.

  \begin{enumerate}
    \item Consider a cap-and-trade system that aims for a total reduction of $21$ units of emis-
    sions is necessary. 
    \begin{enumerate}
      \item What is the equilibrium allocation of permits to each firm?
      \item At what price would these permits sell for at an auction
    \end{enumerate}

    \item Assume that the control authority wanted to reach its objective by using an emissions
    charge system instead.
    \begin{enumerate}
      \item What tax amount should them impose to reach this equilibrium?
      \item How much revenue would the government collect?
    \end{enumerate}
  \end{enumerate}
\end{enumerate}

\end{document}
