\documentclass[11pt]{article}
\title{Math Practice 1}
%% Language and font encodings
\usepackage[english]{babel}
\usepackage[utf8x]{inputenc}
\usepackage[T1]{fontenc}

\usepackage{helvet}

%% Sets page size and margins
\usepackage[letterpaper,top=3cm,bottom=2cm,left=3cm,right=3cm,marginparwidth=1.75cm]{geometry}

%% Useful packages
\usepackage{amsmath}
\usepackage{graphicx}
\usepackage{tcolorbox}
\usepackage{amssymb}
\usepackage{amsthm}
\usepackage{lastpage}
\usepackage{accents}
\usepackage{multicol}

% For better list numbering
\usepackage[shortlabels]{enumitem}

% Font
% \usepackage{tgbonum}


% Tikz
\usepackage{tikz}

\usetikzlibrary{calc,fit,shapes.misc,backgrounds}
\usepackage{pgfplots}
\pgfplotsset{compat = newest}
\usetikzlibrary{positioning, arrows.meta}
\usepgfplotslibrary{fillbetween}

% Headers
\usepackage{fancyhdr}
\pagestyle{fancy}

% Store \@title as \thetitle
\makeatletter
\let\thetitle\@title
\makeatother

\fancyhf{}
\lhead{\fontfamily{qbk}\fontsize{10}{11}\selectfont ECON 3535}
\rhead{\fontfamily{qbk}\fontsize{10}{11}\selectfont \thetitle}
\rfoot{\fontfamily{qbk}\fontsize{10}{11}\selectfont \thepage}


% Sections and Subsections

% define colors
\definecolor{buff-gold}{HTML}{CFB87C}
\definecolor{buff-grey}{HTML}{565A5C}
% custom tcolorbox
\tcbset{colframe=buff-gold, colback=white!100!black}

% new page per section
\usepackage{titlesec}
\newcommand{\sectionbreak}{\clearpage}
% change style of section
\usepackage{sectsty}
\sectionfont{\color{buff-gold} \fontfamily{qbk}\selectfont}
\subsectionfont{\color{buff-grey} \fontfamily{qbk}\selectfont}
\subsubsectionfont{\color{buff-grey} \fontfamily{qbk}\selectfont}



\newtoggle{INCLUDEANSWERS}
\toggletrue{INCLUDEANSWERS}
% \togglefalse{INCLUDEANSWERS}

\newcommand{\answer}[1]{\iftoggle{INCLUDEANSWERS}{{\color{violet!70!white}\textbf{Solution:} #1}}{}}

\newtoggle{INCLUDEPOINTS}
\toggletrue{INCLUDEPOINTS}
\newcommand{\points}[1]{\iftoggle{INCLUDEPOINTS}{{\color{blue!70!white}(#1 pts.)}}{}}


\begin{document}

\subsection*{Two period model of a non-renewable resource}

\begin{enumerate}
  \item Consider extraction of a non-renewable natural resource. The inverse demand function for the depletable resource is $P = 20 - 2Q$ in both periods $1$ and $2$ and the marginal cost of supplying it is $\$3$. The discount rate is $10\%$. There are $10$ units total.

  \begin{enumerate}
    \item Explain what the resource constraint is and then write it in mathematical form

    \item What is the per-unit profit for extracting resources in period 1? 
    
    \item What is the per-unit profit for extracting resources in period 2? What is the present value of the per-unit profit from the viewpoint of period 1?

    \item Describe in words why you must equate the marginal unit's profit in each periods (the optimality condition). 
    
    \item What two pieces of information are needed in order to solve for the optimal extraction $Q_1^*$ and $Q_2^*$?
    
    \item Solve for the optimal extraction $Q_1^*$ and $Q_2^*$.
    
    \item Describe using specific numbers why $Q_1 = 3$ and $Q_2 = 4.5$ is not optimal

    \item What is the marginal user cost? Interpret this number.

    \item Now assume $r = 0$. What is the optimal allocation now? Why did optimal allocation change in the direction that it did?
  \end{enumerate}

  \item Consider extraction of a non-renewable natural resource. The inverse demand function for the depletable resource is $P = 12 - Q$ in both periods $1$ and $2$ and the marginal cost of supplying it is $\$3$. The discount rate is $10\%$. There are $7.5$ units total.

  \begin{enumerate}
    \item Find the equilibrium allocation in each period for resource extraction

    \item Describe using the concept of marignal user cost why $Q_1 = 3$ and $Q_2 = 4.5$ is not optimal

    \item What is the marginal user cost? Interpret this number.

    \item Now assume $r = 0$. What is the optimal allocation now? Why did it change in the direction that it did?
  \end{enumerate}

  \item Consider extraction of a non-renewable natural resource. The inverse demand function for the depletable resource is $P = 12 - Q$ in both periods $1$ and $2$ and the marginal cost of supplying it is $2 + Q/2$. The discount rate is $6\%$. There are $15\$$ units total.

  \begin{enumerate}
    \item Find the equilibrium allocation in each period for resource extraction

    \item What is the marginal user cost? Interpret this number.
  \end{enumerate}
\end{enumerate}

\end{document}
