\documentclass[11pt]{article}
\title{Math Practice 1}
\input{preamble.tex}

\newtoggle{INCLUDEANSWERS}
\toggletrue{INCLUDEANSWERS}
% \togglefalse{INCLUDEANSWERS}

\newcommand{\answer}[1]{\iftoggle{INCLUDEANSWERS}{{\color{violet!70!white}\textbf{Solution:} #1}}{}}

\newtoggle{INCLUDEPOINTS}
\toggletrue{INCLUDEPOINTS}
\newcommand{\points}[1]{\iftoggle{INCLUDEPOINTS}{{\color{blue!70!white}(#1 pts.)}}{}}


\begin{document}

\subsection*{Two period model of a non-renewable resource}

Consider extraction of a non-renewable natural resource. The inverse demand function for the depletable resource is $P = 20 - 2Q$ in both periods $1$ and $2$ and the marginal cost of supplying it is $\$3$. The discount rate is $10\%$. There are $10$ units total.

\begin{enumerate}
  \item Explain what the resource constraint is and then write it in mathematical form

  \item What is the per-unit profit for extracting resources in period 1? 
  
  \item What is the per-unit profit for extracting resources in period 2? What is the present value of the per-unit profit from the viewpoint of period 1?

  \item Describe in words why you must equate the marginal unit's profit in each periods (the optimality condition). 
  
  \item What two pieces of information are needed in order to solve for the optimal extraction $Q_1^*$ and $Q_2^*$?
  
  \item Solve for the optimal extraction $Q_1^*$ and $Q_2^*$.
  
  \item Describe using specific numbers why $Q_1 = 3$ and $Q_2 = 4.5$ is not optimal

  \item What is the marginal user cost? Interpret this number.

  \item Now assume $r = 0$. What is the optimal allocation now? Why did optimal allocation change in the direction that it did?
\end{enumerate}

\newpage
\subsection*{Tradable Permits}

Two firms can control emissions at the following marginal costs: $MC_1 = 80a_x$ and $MC_2 = 40 a_y$ where $a_x$ and $a_y$ are, respectively, the amount of emissions reduced by the first and second firms. Assume that with no control at all, each firm would be emitting $50$ units of emissions or a total of $100$ units for both firms.

\begin{enumerate}
  \item Which firm is better at abating pollution?
  
  \item If the goal is to reduce total emissions to $60$ units. How many units must be abated? Write out the abatement constraint in mathematical terms
  
  \item Consider a uniform standard. How many units must be abated by both firms? How much did each firm have to pay to abate their marginal unit of pollution?

  \item Consider a cap-and-trade system that aims for a total $60$ units of emissions. 
  \begin{enumerate}
    \item In words, describe why the marginal abatement costs for each firm must be equal to eachother in order to be at equilibrium (the optimality condition). 
    
    \item Using the optimality condition and the abatement constraint, solve for the equilibrium allocation of permits to each firm?

    \item At what price would these permits sell for at an auction?
  \end{enumerate}

  \item Assume that the control authority wanted to reach its objective by using an emissions charge system instead.
  \begin{enumerate}
    \item What tax amount should them impose to reach this equilibrium?
    
    \item How much revenue would the government collect?
  \end{enumerate}

  \item Why is cap-and-trade more cost-effective than a uniform standard where each firm reduces pollution by the same amount?
\end{enumerate}

\end{document}
